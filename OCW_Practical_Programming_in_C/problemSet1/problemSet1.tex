% --------------------------------------------------------------
% This is all preamble stuff that you don't have to worry about.
% Head down to where it says "Start here"
% --------------------------------------------------------------
 
\documentclass[12pt]{article}
 
\usepackage[margin=1in]{geometry} 
\usepackage{amsmath,amsthm,amssymb}
\usepackage{hyperref}
\usepackage{listings}
 
\newcommand{\N}{\mathbb{N}}
\newcommand{\Z}{\mathbb{Z}}
 
\newenvironment{theorem}[2][Theorem]{\begin{trivlist}
\item[\hskip \labelsep {\bfseries #1}\hskip \labelsep {\bfseries #2.}]}{\end{trivlist}}
\newenvironment{lemma}[2][Lemma]{\begin{trivlist}
\item[\hskip \labelsep {\bfseries #1}\hskip \labelsep {\bfseries #2.}]}{\end{trivlist}}
\newenvironment{exercise}[2][Exercise]{\begin{trivlist}
\item[\hskip \labelsep {\bfseries #1}\hskip \labelsep {\bfseries #2.}]}{\end{trivlist}}
\newenvironment{problem}[2][Problem]{\begin{trivlist}
\item[\hskip \labelsep {\bfseries #1}\hskip \labelsep {\bfseries #2.}]}{\end{trivlist}}
\newenvironment{question}[2][Question]{\begin{trivlist}
\item[\hskip \labelsep {\bfseries #1}\hskip \labelsep {\bfseries #2.}]}{\end{trivlist}}
\newenvironment{corollary}[2][Corollary]{\begin{trivlist}
\item[\hskip \labelsep {\bfseries #1}\hskip \labelsep {\bfseries #2.}]}{\end{trivlist}}
 
\begin{document}
 
% --------------------------------------------------------------
%                         Start here
% --------------------------------------------------------------
 
\title{Problem set 1}%replace X with the appropriate number
\author{Ingolf D. Petersen\\ %replace with your name
MIT OpenCourseWare \\
6.087 Practical Programming in C} %if necessary, replace with your course title
 
\maketitle
\section*{Description}
Solutions to problem set 1 from the course 
\href{http://ocw.mit.edu/courses/electrical-engineering-and-computer-science/6-087-practical-programming-in-c-january-iap-2010/index.htm}{"Practical Programming in C" from MIT's OpenCourseWare}. I do not own any of the teaching materials presented. Materials used under the creative commons license (see \href{http://ocw.mit.edu/terms/}{http://ocw.mit.edu/terms/}).


\section*{Problem 1.1}
\subsection*{a)}
Curly braces in C group a block of code together. They can be considered the beginning and ending of such a block. Variables defined within a block of code only exist within that block.

\subsection*{b)}
\begin{itemize}
\item \textbf{7}: The number seven. Integer or another numerical type.
\item \textbf{"7"}: A null terminated array of characters.
\item \textbf{'7'}: The character 7.
\end{itemize}

\subsection*{c)}
\begin{lstlisting}[frame=single, language=C]
double ans = 10.0+2.0/((3.0-2.0)*2.0);
\end{lstlisting}

%%%%%%%%%%%%%%
% Problem 1.2
%%%%%%%%%%%%%%
\section*{Problem 1.2}
\begin{verbatim}
Consider the statement
double ans = 18.0/squared(2+1);
For each of the four versions of the function macro squared() below, 
write the corresponding value of ans.
1.  #define  squared(x)  x*x
2.  #define  squared(x)  (x*x)
3.  #define  squared(x)  (x)*(x)
4.  #define  squared(x)  ((x)*(x))
\end{verbatim}

\subsection*{1}
\begin{lstlisting}[frame=single, language=C]
double ans = 18.0/squared(2+1);
double ans = 18.0/2+1*2+1;
double ans = 9.0+2+1;
double ans = 12.0;
\end{lstlisting}

\subsection*{2}
\begin{lstlisting}[frame=single, language=C]
double ans = 18.0/squared(2+1);
double ans = 18.0/(2+1*2+1);
double ans = 18.0/(2+2+1);
double ans = 18.0/5;
double ans = 3.6;
\end{lstlisting}

\subsection*{3}
\begin{lstlisting}[frame=single, language=C]
double ans = 18.0/squared(2+1);
double ans = 18.0/(2+1)*(2+1);
double ans = 18.0/3*3;
double ans = 6.0*3;
double ans = 18.0;
\end{lstlisting}

\subsection*{4}
\begin{lstlisting}[frame=single, language=C]
double ans = 18.0/squared(2+1);
double ans = 18.0/((2+1)*(2+1));
double ans = 18.0/((3)*(3));
double ans = 18.0/(3*3);
double ans = 18.0/(9);
double ans = 18.0/9;
double ans = 2.0;
\end{lstlisting}

\section*{Problem 1.3}

\begin{lstlisting}[frame=single, language=bash, breaklines=true]
ingodav@ingodav-PC /cygdrive/e/Dropbox/courses/online_courses/OCW_Practical_Programming_in_C/problemSet1
$ gcc  -g  -O0  -Wall  hello.c  -o hello.o

ingodav@ingodav-PC /cygdrive/e/Dropbox/courses/online_courses/OCW_Practical_Programming_in_C/problemSet1
$ gdb hello.o
GNU gdb (GDB) 7.6.50.20130728-cvs (cygwin-special)
Copyright (C) 2013 Free Software Foundation, Inc.
License GPLv3+: GNU GPL version 3 or later <http://gnu.org/licenses/gpl.html>
This is free software: you are free to change and redistribute it.
There is NO WARRANTY, to the extent permitted by law.  Type "show copying"
and "show warranty" for details.
This GDB was configured as "i686-pc-cygwin".
Type "show configuration" for configuration details.
For bug reporting instructions, please see:
<http://www.gnu.org/software/gdb/bugs/>.
Find the GDB manual and other documentation resources online at:
<http://www.gnu.org/software/gdb/documentation/>.
For help, type "help".
Type "apropos word" to search for commands related to "word".
..
Reading symbols from /cygdrive/e/Dropbox/courses/online_courses/OCW_Practical_Programming_in_C/problemSet1/hello.o...done.
(gdb) run
Starting program: /cygdrive/e/Dropbox/courses/online_courses/OCW_Practical_Programming_in_C/problemSet1/hello.o
[New Thread 3832.0x2440]
[New Thread 3832.0x20bc]
hello,  6.087  students
[Inferior 1 (process 3832) exited normally]
(gdb) quit

ingodav@ingodav-PC /cygdrive/e/Dropbox/courses/online_courses/OCW_Practical_Programming_in_C/problemSet1
$
\end{lstlisting}

\section*{Problem 1.4}
\lstinputlisting[language=C, frame=single]{base.c}

\section*{Problem 1.5}
\subsection*{a)}
Includes do not need a semicolon to end the statement. Correct version:
\begin{lstlisting}[frame=single, language=bash, breaklines=true]
#include <stdio.h>
\end{lstlisting}

\subsection*{b)}
Type mismatch in function definition. The function returns an integer but accepts a void input argument. Correct version:

\begin{lstlisting}[frame=single, language=bash, breaklines=true]
int function(int arg1)
{
  return  arg1-1;
}
\end{lstlisting}

\subsection*{c)}
The assignment operator should not be used when using \#define.
\begin{lstlisting}[frame=single, language=bash, breaklines=true]
#define MESSAGE "Happy  new  year!"
puts(MESSAGE);
\end{lstlisting}



 
% --------------------------------------------------------------
%     You don't have to mess with anything below this line.
% --------------------------------------------------------------
 
\end{document}