% --------------------------------------------------------------
% This is all preamble stuff that you don't have to worry about.
% Head down to where it says "Start here"
% --------------------------------------------------------------
 
\documentclass[12pt]{article}
 
\usepackage[margin=1in]{geometry} 
\usepackage{amsmath,amsthm,amssymb}
\usepackage{hyperref}
\usepackage{listings}
 
\newcommand{\N}{\mathbb{N}}
\newcommand{\Z}{\mathbb{Z}}
 
\newenvironment{theorem}[2][Theorem]{\begin{trivlist}
\item[\hskip \labelsep {\bfseries #1}\hskip \labelsep {\bfseries #2.}]}{\end{trivlist}}
\newenvironment{lemma}[2][Lemma]{\begin{trivlist}
\item[\hskip \labelsep {\bfseries #1}\hskip \labelsep {\bfseries #2.}]}{\end{trivlist}}
\newenvironment{exercise}[2][Exercise]{\begin{trivlist}
\item[\hskip \labelsep {\bfseries #1}\hskip \labelsep {\bfseries #2.}]}{\end{trivlist}}
\newenvironment{problem}[2][Problem]{\begin{trivlist}
\item[\hskip \labelsep {\bfseries #1}\hskip \labelsep {\bfseries #2.}]}{\end{trivlist}}
\newenvironment{question}[2][Question]{\begin{trivlist}
\item[\hskip \labelsep {\bfseries #1}\hskip \labelsep {\bfseries #2.}]}{\end{trivlist}}
\newenvironment{corollary}[2][Corollary]{\begin{trivlist}
\item[\hskip \labelsep {\bfseries #1}\hskip \labelsep {\bfseries #2.}]}{\end{trivlist}}
 
\begin{document}
 
% --------------------------------------------------------------
%                         Start here
% --------------------------------------------------------------
 
\title{Problem set 2}%replace X with the appropriate number
\author{Ingolf D. Petersen\\ %replace with your name
MIT OpenCourseWare \\
6.087 Practical Programming in C} %if necessary, replace with your course title
 
\maketitle
\section*{Description}
Solutions to problem set 2 from the course 
\href{http://ocw.mit.edu/courses/electrical-engineering-and-computer-science/6-087-practical-programming-in-c-january-iap-2010/index.htm}{"Practical Programming in C" from MIT's OpenCourseWare}. I do not own any of the teaching materials presented. Materials used under the creative commons license (see \href{http://ocw.mit.edu/terms/}{http://ocw.mit.edu/terms/}).


\section*{Problem 2.1}
On a 64 bit machine the following results are obtained. NOTE: Using 32 bit Cygwin.

\begin{table}[h!tb]
	\centering
    \begin{tabular}{|c|c|}
    \hline
    \textbf{Type}          & \textbf{Size (bytes)} \\ \hline
    char          & 1            \\
    unsigned char & 1            \\
    short         & 2            \\
    int           & 4            \\
    unsigned int  & 4            \\
    unsigned long & 4            \\
    float         & 4            \\
    \hline
    \end{tabular}
\end{table}

\lstinputlisting[language=C, frame=single]{p1.c}

\section*{Problem 2.2}



%\subsection*{b)}
%\begin{itemize}
%\item \textbf{7}: The number seven. Integer or another numerical type.
%\item \textbf{"7"}: A null terminated array of characters.
%\item \textbf{'7'}: The character 7.
%\end{itemize}
%
%\subsection*{c)}
%\begin{lstlisting}[frame=single, language=C]
%double ans = 10.0+2.0/((3.0-2.0)*2.0);
%\end{lstlisting}

%%%%%%%%%%%%%%
% Problem 1.2
%%%%%%%%%%%%%%



 
% --------------------------------------------------------------
%     You don't have to mess with anything below this line.
% --------------------------------------------------------------
 
\end{document}