% --------------------------------------------------------------
% This is all preamble stuff that you don't have to worry about.
% Head down to where it says "Start here"
% --------------------------------------------------------------
 
\documentclass[12pt]{article}
 
\usepackage[margin=1in]{geometry} 
\usepackage{amsmath,amsthm,amssymb}
\usepackage{hyperref}
\usepackage{listings}
 
\newcommand{\N}{\mathbb{N}}
\newcommand{\Z}{\mathbb{Z}}
 
\newenvironment{theorem}[2][Theorem]{\begin{trivlist}
\item[\hskip \labelsep {\bfseries #1}\hskip \labelsep {\bfseries #2.}]}{\end{trivlist}}
\newenvironment{lemma}[2][Lemma]{\begin{trivlist}
\item[\hskip \labelsep {\bfseries #1}\hskip \labelsep {\bfseries #2.}]}{\end{trivlist}}
\newenvironment{exercise}[2][Exercise]{\begin{trivlist}
\item[\hskip \labelsep {\bfseries #1}\hskip \labelsep {\bfseries #2.}]}{\end{trivlist}}
\newenvironment{problem}[2][Problem]{\begin{trivlist}
\item[\hskip \labelsep {\bfseries #1}\hskip \labelsep {\bfseries #2.}]}{\end{trivlist}}
\newenvironment{question}[2][Question]{\begin{trivlist}
\item[\hskip \labelsep {\bfseries #1}\hskip \labelsep {\bfseries #2.}]}{\end{trivlist}}
\newenvironment{corollary}[2][Corollary]{\begin{trivlist}
\item[\hskip \labelsep {\bfseries #1}\hskip \labelsep {\bfseries #2.}]}{\end{trivlist}}
 
\begin{document}
 
% --------------------------------------------------------------
%                         Start here
% --------------------------------------------------------------
 
\title{Weekly Homework X}%replace X with the appropriate number
\author{Tony Stark\\ %replace with your name
MIT OpenCourseWare \\
Practical Programming in C} %if necessary, replace with your course title
 
\maketitle
\section*{Description}
Solutions to problem set 1 from the course 
\href{http://ocw.mit.edu/courses/electrical-engineering-and-computer-science/6-087-practical-programming-in-c-january-iap-2010/index.htm}{"Practical Programming in C" from MIT's OpenCourseWare}.


\section*{Problem 1.1}
\subsection*{a)}
Curly braces in C group a block of code together. They can be considered the beginning and ending of such a block. Variables defined within a block of code only exist within that block.

\subsection*{b)}
\begin{itemize}
\item \textbf{7}: The number seven. Integer or another numerical type.
\item \textbf{"7"}: A null terminated array of characters.
\item \textbf{'7'}: The character 7.
\end{itemize}

\subsection*{c)}
\begin{lstlisting}[frame=single, language=C]
double ans = 10.0+2.0/((3.0-2.0)*2.0);
\end{lstlisting}

%%%%%%%%%%%%%%
% Problem 1.2
%%%%%%%%%%%%%%
\section*{Problem 1.2}
\begin{verbatim}
Consider the statement
double ans = 18.0/squared(2+1);
For each of the four versions of the function macro squared() below, 
write the corresponding value of ans.
1.  #define  squared(x)  x*x
2.  #define  squared(x)  (x*x)
3.  #define  squared(x)  (x)*(x)
4.  #define  squared(x)  ((x)*(x))
\end{verbatim}

\subsection*{1}
\begin{lstlisting}[frame=single, language=C]
double ans = 18.0/squared(2+1);
double ans = 18.0/2+1*2+1;
double ans = 9.0+2+1;
double ans = 12.0;
\end{lstlisting}

\subsection*{2}
\begin{lstlisting}[frame=single, language=C]
double ans = 18.0/squared(2+1);
double ans = 18.0/(2+1*2+1);
double ans = 18.0/(2+2+1);
double ans = 18.0/5;
double ans = 3.6;
\end{lstlisting}

\subsection*{3}
\begin{lstlisting}[frame=single, language=C]
double ans = 18.0/squared(2+1);
double ans = 18.0/(2+1)*(2+1);
double ans = 18.0/3*3;
double ans = 6.0*3;
double ans = 18.0;
\end{lstlisting}

\subsection*{4}
\begin{lstlisting}[frame=single, language=C]
double ans = 18.0/squared(2+1);
double ans = 18.0/((2+1)*(2+1));
double ans = 18.0/((3)*(3));
double ans = 18.0/(3*3);
double ans = 18.0/(9);
double ans = 18.0/9;
double ans = 2.0;
\end{lstlisting}

\section*{Problem 1.3}
 
% --------------------------------------------------------------
%     You don't have to mess with anything below this line.
% --------------------------------------------------------------
 
\end{document}